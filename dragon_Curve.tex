% This LaTeX was auto-generated from MATLAB code.
% To make changes, update the MATLAB code and export to LaTeX again.

\documentclass{article}

\usepackage[utf8]{inputenc}
\usepackage[T1]{fontenc}
\usepackage{lmodern}
\usepackage{graphicx}
\usepackage{color}
\usepackage{hyperref}
\usepackage{amsmath}
\usepackage{amsfonts}
\usepackage{epstopdf}
\usepackage[table]{xcolor}
\usepackage{matlab}

\sloppy
\epstopdfsetup{outdir=./}
\graphicspath{ {./dragon_Curve_images/} }

\begin{document}

\begin{matlabcode}
StrX=' X+YF+';
StrY='-FX-Y';
Str1=strrep(strrep(strrep(StrX,'X','1'),'Y',StrY),'1',StrX);
Str2=strrep(strrep(strrep(StrX,'X','1'),'Y',StrY),'1',StrX);
Str3=strrep(strrep(strrep(Str2,'X','1'),'Y',StrY),'1',StrX);
Str4=strrep(strrep(strrep(Str3,'X','1'),'Y',StrY),'1',StrX);
Str5=strrep(strrep(strrep(Str4,'X','1'),'Y',StrY),'1',StrX);
Str6=strrep(strrep(strrep(Str5,'X','1'),'Y',StrY),'1',StrX);
Str7=strrep(strrep(strrep(Str6,'X','1'),'Y',StrY),'1',StrX);
n=length(Str1)
\end{matlabcode}
\begin{matlaboutput}
n = 15
\end{matlaboutput}
\begin{matlabcode}
index=0;
% following three arrays are to track branching in the tree
% we do not use stack
posx(1)=0; %Initialize. Also tell matlab posx is an array
posy(1)=0; %Initialize. Also tell matlab posy is an array
posa(1)=0;%Initialize. Also tell matlab posa is an array
% following variables are for drawing line segment
oldx=0;
oldy=0;
newx=0;
newy=0;
% after plotting a segment, newx and newy become oldx and oldy
alpha=0;
delta=90;
figure
for i=1:n
comd=Str1(i);
switch(comd)
case 'F'
newx=oldx+cos(deg2rad(alpha));
newy=oldy+sin(deg2rad(alpha));
line([oldx newx],[oldy newy],'Color','R','LineStyle','-','LineWidth',1);
drawnow;
oldx=newx;
oldy=newy;
case '+'
alpha=alpha+delta;
case '-'
alpha=alpha-delta;
end
end
\end{matlabcode}
\begin{center}
\includegraphics[width=\maxwidth{56.196688409433015em}]{figure_0.png}
\end{center}


\begin{matlabcode}
StrX=' X+YF+';
StrY='-FX-Y';
Str1=strrep(strrep(strrep(StrX,'X','1'),'Y',StrY),'1',StrX);
Str2=strrep(strrep(strrep(StrX,'X','1'),'Y',StrY),'1',StrX);
Str3=strrep(strrep(strrep(Str2,'X','1'),'Y',StrY),'1',StrX);
Str4=strrep(strrep(strrep(Str3,'X','1'),'Y',StrY),'1',StrX);
Str5=strrep(strrep(strrep(Str4,'X','1'),'Y',StrY),'1',StrX);
Str6=strrep(strrep(strrep(Str5,'X','1'),'Y',StrY),'1',StrX);
Str7=strrep(strrep(strrep(Str6,'X','1'),'Y',StrY),'1',StrX);
n=length(Str3)
\end{matlabcode}
\begin{matlaboutput}
n = 33
\end{matlaboutput}
\begin{matlabcode}
index=0;
% following three arrays are to track branching in the tree
% we do not use stack
posx(1)=0; %Initialize. Also tell matlab posx is an array
posy(1)=0; %Initialize. Also tell matlab posy is an array
posa(1)=0;%Initialize. Also tell matlab posa is an array
% following variables are for drawing line segment
oldx=0;
oldy=0;
newx=0;
newy=0;
% after plotting a segment, newx and newy become oldx and oldy
alpha=0;
delta=90;
figure
for i=1:n
comd=Str3(i);
switch(comd)
case 'F'
newx=oldx+cos(deg2rad(alpha));
newy=oldy+sin(deg2rad(alpha));
line([oldx newx],[oldy newy],'Color','black','LineStyle','-','LineWidth',1);
drawnow;
oldx=newx;
oldy=newy;
case '+'
alpha=alpha+delta;
case '-'
alpha=alpha-delta;
end
end
\end{matlabcode}
\begin{center}
\includegraphics[width=\maxwidth{56.196688409433015em}]{figure_1.png}
\end{center}


\begin{matlabcode}
StrX=' X+YF+';
StrY='-FX-Y';
Str1=strrep(strrep(strrep(StrX,'X','1'),'Y',StrY),'1',StrX);
Str2=strrep(strrep(strrep(StrX,'X','1'),'Y',StrY),'1',StrX);
Str3=strrep(strrep(strrep(Str2,'X','1'),'Y',StrY),'1',StrX);
Str4=strrep(strrep(strrep(Str3,'X','1'),'Y',StrY),'1',StrX);
Str5=strrep(strrep(strrep(Str4,'X','1'),'Y',StrY),'1',StrX);
Str6=strrep(strrep(strrep(Str5,'X','1'),'Y',StrY),'1',StrX);
Str7=strrep(strrep(strrep(Str6,'X','1'),'Y',StrY),'1',StrX);
n=length(Str4)
\end{matlabcode}
\begin{matlaboutput}
n = 69
\end{matlaboutput}
\begin{matlabcode}
index=0;
% following three arrays are to track branching in the tree
% we do not use stack
posx(1)=0; %Initialize. Also tell matlab posx is an array
posy(1)=0; %Initialize. Also tell matlab posy is an array
posa(1)=0;%Initialize. Also tell matlab posa is an array
% following variables are for drawing line segment
oldx=0;
oldy=0;
newx=0;
newy=0;
% after plotting a segment, newx and newy become oldx and oldy
alpha=0;
delta=90;
figure
for i=1:n
comd=Str4(i);
switch(comd)
case 'F'
newx=oldx+cos(deg2rad(alpha));
newy=oldy+sin(deg2rad(alpha));
line([oldx newx],[oldy newy],'Color','black','LineStyle','-','LineWidth',1);
drawnow;
oldx=newx;
oldy=newy;
case '+'
alpha=alpha+delta;
case '-'
alpha=alpha-delta;
end
end
\end{matlabcode}
\begin{center}
\includegraphics[width=\maxwidth{56.196688409433015em}]{figure_2.png}
\end{center}


\begin{matlabcode}
StrX=' X+YF+';
StrY='-FX-Y';
Str1=strrep(strrep(strrep(StrX,'X','1'),'Y',StrY),'1',StrX);
Str2=strrep(strrep(strrep(StrX,'X','1'),'Y',StrY),'1',StrX);
Str3=strrep(strrep(strrep(Str2,'X','1'),'Y',StrY),'1',StrX);
Str4=strrep(strrep(strrep(Str3,'X','1'),'Y',StrY),'1',StrX);
Str5=strrep(strrep(strrep(Str4,'X','1'),'Y',StrY),'1',StrX);
Str6=strrep(strrep(strrep(Str5,'X','1'),'Y',StrY),'1',StrX);
Str7=strrep(strrep(strrep(Str6,'X','1'),'Y',StrY),'1',StrX);
n=length(Str5)
\end{matlabcode}
\begin{matlaboutput}
n = 141
\end{matlaboutput}
\begin{matlabcode}
index=0;
% following three arrays are to track branching in the tree
% we do not use stack
posx(1)=0; %Initialize. Also tell matlab posx is an array
posy(1)=0; %Initialize. Also tell matlab posy is an array
posa(1)=0;%Initialize. Also tell matlab posa is an array
% following variables are for drawing line segment
oldx=0;
oldy=0;
newx=0;
newy=0;
% after plotting a segment, newx and newy become oldx and oldy
alpha=0;
delta=90;
figure
for i=1:n
comd=Str5(i);
switch(comd)
case 'F'
newx=oldx+cos(deg2rad(alpha));
newy=oldy+sin(deg2rad(alpha));
line([oldx newx],[oldy newy],'Color','black','LineStyle','-','LineWidth',1);
drawnow;
oldx=newx;
oldy=newy;
case '+'
alpha=alpha+delta;
case '-'
alpha=alpha-delta;
end
end
\end{matlabcode}
\begin{center}
\includegraphics[width=\maxwidth{56.196688409433015em}]{figure_3.png}
\end{center}


\begin{matlabcode}
StrX=' X+YF+';
StrY='-FX-Y';
Str1=strrep(strrep(strrep(StrX,'X','1'),'Y',StrY),'1',StrX);
Str2=strrep(strrep(strrep(StrX,'X','1'),'Y',StrY),'1',StrX);
Str3=strrep(strrep(strrep(Str2,'X','1'),'Y',StrY),'1',StrX);
Str4=strrep(strrep(strrep(Str3,'X','1'),'Y',StrY),'1',StrX);
Str5=strrep(strrep(strrep(Str4,'X','1'),'Y',StrY),'1',StrX);
Str6=strrep(strrep(strrep(Str5,'X','1'),'Y',StrY),'1',StrX);
Str7=strrep(strrep(strrep(Str6,'X','1'),'Y',StrY),'1',StrX);
n=length(Str6)
\end{matlabcode}
\begin{matlaboutput}
n = 285
\end{matlaboutput}
\begin{matlabcode}
index=0;
% following three arrays are to track branching in the tree
% we do not use stack
posx(1)=0; %Initialize. Also tell matlab posx is an array
posy(1)=0; %Initialize. Also tell matlab posy is an array
posa(1)=0;%Initialize. Also tell matlab posa is an array
% following variables are for drawing line segment
oldx=0;
oldy=0;
newx=0;
newy=0;
% after plotting a segment, newx and newy become oldx and oldy
alpha=0;
delta=90;
figure
for i=1:n
comd=Str6(i);
switch(comd)
case 'F'
newx=oldx+cos(deg2rad(alpha));
newy=oldy+sin(deg2rad(alpha));
line([oldx newx],[oldy newy],'Color','black','LineStyle','-','LineWidth',1);
drawnow;
oldx=newx;
oldy=newy;
case '+'
alpha=alpha+delta;
case '-'
alpha=alpha-delta;
end
end
\end{matlabcode}
\begin{center}
\includegraphics[width=\maxwidth{56.196688409433015em}]{figure_4.png}
\end{center}


\begin{matlabcode}
StrX=' X+YF+';
StrY='-FX-Y';
Str1=strrep(strrep(strrep(StrX,'X','1'),'Y',StrY),'1',StrX);
Str2=strrep(strrep(strrep(StrX,'X','1'),'Y',StrY),'1',StrX);
Str3=strrep(strrep(strrep(Str2,'X','1'),'Y',StrY),'1',StrX);
Str4=strrep(strrep(strrep(Str3,'X','1'),'Y',StrY),'1',StrX);
Str5=strrep(strrep(strrep(Str4,'X','1'),'Y',StrY),'1',StrX);
Str6=strrep(strrep(strrep(Str5,'X','1'),'Y',StrY),'1',StrX);
Str7=strrep(strrep(strrep(Str6,'X','1'),'Y',StrY),'1',StrX);
n=length(Str7)
\end{matlabcode}
\begin{matlaboutput}
n = 573
\end{matlaboutput}
\begin{matlabcode}
index=0;
% following three arrays are to track branching in the tree
% we do not use stack
posx(1)=0; %Initialize. Also tell matlab posx is an array
posy(1)=0; %Initialize. Also tell matlab posy is an array
posa(1)=0;%Initialize. Also tell matlab posa is an array
% following variables are for drawing line segment
oldx=0;
oldy=0;
newx=0;
newy=0;
% after plotting a segment, newx and newy become oldx and oldy
alpha=0;
delta=90;
figure
for i=1:n
comd=Str7(i);
switch(comd)
case 'F'
newx=oldx+cos(deg2rad(alpha));
newy=oldy+sin(deg2rad(alpha));
line([oldx newx],[oldy newy],'Color','black','LineStyle','-','LineWidth',1);
drawnow;
oldx=newx;
oldy=newy;
case '+'
alpha=alpha+delta;
case '-'
alpha=alpha-delta;
end
end
\end{matlabcode}
\begin{center}
\includegraphics[width=\maxwidth{56.196688409433015em}]{figure_5.png}
\end{center}


\begin{matlabcode}
StrX=' X+YF+';
StrY='-FX-Y';
Str1=strrep(strrep(strrep(StrX,'X','1'),'Y',StrY),'1',StrX);
Str2=strrep(strrep(strrep(StrX,'X','1'),'Y',StrY),'1',StrX);
Str3=strrep(strrep(strrep(Str2,'X','1'),'Y',StrY),'1',StrX);
Str4=strrep(strrep(strrep(Str3,'X','1'),'Y',StrY),'1',StrX);
Str5=strrep(strrep(strrep(Str4,'X','1'),'Y',StrY),'1',StrX);
Str6=strrep(strrep(strrep(Str5,'X','1'),'Y',StrY),'1',StrX);
Str7=strrep(strrep(strrep(Str6,'X','1'),'Y',StrY),'1',StrX);
Str8=strrep(strrep(strrep(Str7,'X','1'),'Y',StrY),'1',StrX);
n=length(Str8)
\end{matlabcode}
\begin{matlaboutput}
n = 1149
\end{matlaboutput}
\begin{matlabcode}
index=0;
% following three arrays are to track branching in the tree
% we do not use stack
posx(1)=0; %Initialize. Also tell matlab posx is an array
posy(1)=0; %Initialize. Also tell matlab posy is an array
posa(1)=0;%Initialize. Also tell matlab posa is an array
% following variables are for drawing line segment
oldx=0;
oldy=0;
newx=0;
newy=0;
% after plotting a segment, newx and newy become oldx and oldy
alpha=0;
delta=90;
figure
for i=1:n
comd=Str8(i);
switch(comd)
case 'F'
newx=oldx+cos(deg2rad(alpha));
newy=oldy+sin(deg2rad(alpha));
line([oldx newx],[oldy newy],'Color','black','LineStyle','-','LineWidth',1);
drawnow;
oldx=newx;
oldy=newy;
case '+'
alpha=alpha+delta;
case '-'
alpha=alpha-delta;
end
end
\end{matlabcode}
\begin{center}
\includegraphics[width=\maxwidth{56.196688409433015em}]{figure_6.png}
\end{center}


\begin{matlabcode}
StrX=' X+YF+';
StrY='-FX-Y';
Str1=strrep(strrep(strrep(StrX,'X','1'),'Y',StrY),'1',StrX);
Str2=strrep(strrep(strrep(StrX,'X','1'),'Y',StrY),'1',StrX);
Str3=strrep(strrep(strrep(Str2,'X','1'),'Y',StrY),'1',StrX);
Str4=strrep(strrep(strrep(Str3,'X','1'),'Y',StrY),'1',StrX);
Str5=strrep(strrep(strrep(Str4,'X','1'),'Y',StrY),'1',StrX);
Str6=strrep(strrep(strrep(Str5,'X','1'),'Y',StrY),'1',StrX);
Str7=strrep(strrep(strrep(Str6,'X','1'),'Y',StrY),'1',StrX);
Str8=strrep(strrep(strrep(Str7,'X','1'),'Y',StrY),'1',StrX);
Str9=strrep(strrep(strrep(Str8,'X','1'),'Y',StrY),'1',StrX);
n=length(Str9)
\end{matlabcode}
\begin{matlaboutput}
n = 2301
\end{matlaboutput}
\begin{matlabcode}
index=0;
% following three arrays are to track branching in the tree
% we do not use stack
posx(1)=0; %Initialize. Also tell matlab posx is an array
posy(1)=0; %Initialize. Also tell matlab posy is an array
posa(1)=0;%Initialize. Also tell matlab posa is an array
% following variables are for drawing line segment
oldx=0;
oldy=0;
newx=0;
newy=0;
% after plotting a segment, newx and newy become oldx and oldy
alpha=0;
delta=90;
figure
for i=1:n
comd=Str9(i);
switch(comd)
case 'F'
newx=oldx+cos(deg2rad(alpha));
newy=oldy+sin(deg2rad(alpha));
line([oldx newx],[oldy newy],'Color','black','LineStyle','-','LineWidth',1);
drawnow;
oldx=newx;
oldy=newy;
case '+'
alpha=alpha+delta;
case '-'
alpha=alpha-delta;
end
end
\end{matlabcode}
\begin{center}
\includegraphics[width=\maxwidth{56.196688409433015em}]{figure_7.png}
\end{center}


\begin{matlabcode}
StrX=' X+YF+';
StrY='-FX-Y';
Str1=strrep(strrep(strrep(StrX,'X','1'),'Y',StrY),'1',StrX);
Str2=strrep(strrep(strrep(StrX,'X','1'),'Y',StrY),'1',StrX);
Str3=strrep(strrep(strrep(Str2,'X','1'),'Y',StrY),'1',StrX);
Str4=strrep(strrep(strrep(Str3,'X','1'),'Y',StrY),'1',StrX);
Str5=strrep(strrep(strrep(Str4,'X','1'),'Y',StrY),'1',StrX);
Str6=strrep(strrep(strrep(Str5,'X','1'),'Y',StrY),'1',StrX);
Str7=strrep(strrep(strrep(Str6,'X','1'),'Y',StrY),'1',StrX);
Str8=strrep(strrep(strrep(Str7,'X','1'),'Y',StrY),'1',StrX);
Str9=strrep(strrep(strrep(Str8,'X','1'),'Y',StrY),'1',StrX);
Str10=strrep(strrep(strrep(Str9,'X','1'),'Y',StrY),'1',StrX);
n=length(Str10)
\end{matlabcode}
\begin{matlaboutput}
n = 4605
\end{matlaboutput}
\begin{matlabcode}
index=0;
% following three arrays are to track branching in the tree
% we do not use stack
posx(1)=0; %Initialize. Also tell matlab posx is an array
posy(1)=0; %Initialize. Also tell matlab posy is an array
posa(1)=0;%Initialize. Also tell matlab posa is an array
% following variables are for drawing line segment
oldx=0;
oldy=0;
newx=0;
newy=0;
% after plotting a segment, newx and newy become oldx and oldy
alpha=0;
delta=90;
figure
for i=1:n
comd=Str10(i);
switch(comd)
case 'F'
newx=oldx+cos(deg2rad(alpha));
newy=oldy+sin(deg2rad(alpha));
line([oldx newx],[oldy newy],'Color','black','LineStyle','-','LineWidth',1);
drawnow;
oldx=newx;
oldy=newy;
case '+'
alpha=alpha+delta;
case '-'
alpha=alpha-delta;
end
end
\end{matlabcode}
\begin{center}
\includegraphics[width=\maxwidth{56.196688409433015em}]{figure_8.png}
\end{center}


\begin{matlabcode}
StrX=' X+YF+';
StrY='-FX-Y';
Str1=strrep(strrep(strrep(StrX,'X','1'),'Y',StrY),'1',StrX);
Str2=strrep(strrep(strrep(StrX,'X','1'),'Y',StrY),'1',StrX);
Str3=strrep(strrep(strrep(Str2,'X','1'),'Y',StrY),'1',StrX);
Str4=strrep(strrep(strrep(Str3,'X','1'),'Y',StrY),'1',StrX);
Str5=strrep(strrep(strrep(Str4,'X','1'),'Y',StrY),'1',StrX);
Str6=strrep(strrep(strrep(Str5,'X','1'),'Y',StrY),'1',StrX);
Str7=strrep(strrep(strrep(Str6,'X','1'),'Y',StrY),'1',StrX);
Str8=strrep(strrep(strrep(Str7,'X','1'),'Y',StrY),'1',StrX);
Str9=strrep(strrep(strrep(Str8,'X','1'),'Y',StrY),'1',StrX);
Str10=strrep(strrep(strrep(Str9,'X','1'),'Y',StrY),'1',StrX);
Str11=strrep(strrep(strrep(Str10,'X','1'),'Y',StrY),'1',StrX);
n=length(Str11)
\end{matlabcode}
\begin{matlaboutput}
n = 9213
\end{matlaboutput}
\begin{matlabcode}
index=0;
% following three arrays are to track branching in the tree
% we do not use stack
posx(1)=0; %Initialize. Also tell matlab posx is an array
posy(1)=0; %Initialize. Also tell matlab posy is an array
posa(1)=0;%Initialize. Also tell matlab posa is an array
% following variables are for drawing line segment
oldx=0;
oldy=0;
newx=0;
newy=0;
% after plotting a segment, newx and newy become oldx and oldy
alpha=0;
delta=90;
figure
for i=1:n
comd=Str11(i);
switch(comd)
case 'F'
newx=oldx+cos(deg2rad(alpha));
newy=oldy+sin(deg2rad(alpha));
line([oldx newx],[oldy newy],'Color','black','LineStyle','-','LineWidth',1);
drawnow;
oldx=newx;
oldy=newy;
case '+'
alpha=alpha+delta;
case '-'
alpha=alpha-delta;
end
end
\end{matlabcode}
\begin{center}
\includegraphics[width=\maxwidth{56.196688409433015em}]{figure_9.png}
\end{center}


\begin{matlabcode}

\end{matlabcode}

\end{document}
